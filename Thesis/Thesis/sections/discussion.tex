% section
\section{Discussion} \label{section::discussion}
 The related work showed, how differently state-of-the-art architectures for image-/video-prediction get to the same result, but also how they rely on the same
 recurrent module to achieve this results. It also showed how previous work in a field is used to adapt and overcome issues and reach better accuracy.\\
 In this example the paper from Shi et al. \cite{Shi2015}, where they used the previous published work from Srivastava et al. \cite{Srivastava2015}, but invented 
 a novel LSTM module, which used in the Autoencoder architecture outperforms the inital architecture. Both paper showed, how impressive the usage of an 
 Autoencoder architecture is for image-/video-prediction, so possibly Patraucean et al. \cite{Patraucean2015} decided therefore to use a ConvLSTM architecture and 
 an Autoencoder scheme, but additional brought optical flow into the research field, as well as the idea to split the sequence information into space and time. 
 Lotter et al. \cite{Lotter2016} showed, that a concept known from the neuroscience literature can be used in a slightly adapted way,
 using also a LSTM module, to let a neural network perform nearly the same way, a human beeing performs.\\There are several discussions regarding this paper, if
 the authors really re-modeled the idea of predictive coding, this is shortly discussed by Sinapayen and Noda \cite{Sinapayen2019} and very deeply by Rane et al. 
 \cite{Rane2019}. Wang et al. \cite{Wang2017} then invented a novel ConvLSTM module, which not only has temporal memory, but also a spatiotemporal memory.
 \\\\
 %
 The implementation of the networks showed, how important it is nowadays for a machine learning researcher to have a very good understanding of a variety of deep 
 learning frameworks, such as PyTorch \cite{Paszke2019}.\\During the implementation of the networks and the review of relevant sources,
 I read several code bases which
 were not written very comprehensive and adaptable and in several different deep learning frameworks, which sometimes differ very strongly.
 This might be, due to the reason that the code should only work for the researches themselves, the researches use the framework which they are most
 comfortable with and is therefore not described that well, but I believe that this should be done always, especially for people not working everyday with such a 
 programming language or deep learning framework.
 \\\\
 %
 The first experiments, were all hyperparameter were fixed, showed a problem, which is fundamental in machine learning, hyperparameter optimization. Looking at
 the results from the first experiments, one can clearly see, that changing the standard ConvLSTM module with the more advanced PredRNN module didn't
 showed any performance boost.\\In contrast, the change only increased the number of trainable parameter in the networks, which will lead to a worse performance
 of the network. Thinking about embedded systems, with only few resources per task, this is very important, even for neural networks, which are expensive during
 training, but cheap for execution. After performing hyperparameter optimization as last task of the experiments, a simple example already showed a performance
 boost of $> 40\%$. Seeing this results, it would be interesting to also perform hyperparamter optimization for all $18$ setups and perform the tests again to
 see the full results afterwars. This was not unfortunately not applicable due to the limited time of the bachelorthesis.