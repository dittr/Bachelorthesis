% Abstract
The thesis will compare different state-of-the-art solutions for image-/video-prediction \ref{subsection::imageprediction}.
The main module of those solutions, which is the core aspect of this work, is the LSTM (Long short-term memory) \ref{subsection::lstm}.
This module was invented by Hochreiter and Schmidhuber  \cite{Hochreiter1997} in 1997 and is used heavily in the field of image-\& video-prediction since then, e.g. in Srivastava et al. 
\cite{Srivastava2015}.
During the time the module got many different add-ons and changes, which are described in different papers (\cite{Patraucean2015}, \cite{Lotter2016}, \cite{Wang2017}, \cite{Wang2018} and many 
more.). 
This thesis is giving a comprehensive overview over a selection of image-/video-prediction algorithms and experimenting if the more advanced solution PredRNN \cite{Wang2017} is able to outperform the standard ConvLSTM module. This is done, by implementing
three different state-of-the-art solutions for image-/video-prediction (\cite{Shi2015}, \cite{Patraucean2015} and \cite{Lotter2016}).
All of them use a ConvLSTM solution as recurrent sub-module, which is changed during the experiments
with the more advanced solution PredRNN \cite{Wang2017}, to see if the usage of a more advanced recurrent sub-module will increase the accuracy.\\
The algorithms are therefore re-implemented in PyTorch \cite{Paszke2019}.