% section
\section{Introduction} \label{section::introduction}
 This thesis gives an overview of different state-of-the-art solutions for image-/video-prediction and implements different algorithms as baselines to change the 
 used recurrent sub-module with a more advanced recurrent module, to see if a peformance improvement can be achieved.
 \\\\
 Chapter~\ref{section::theory} gives an overview of necessary machine learning theory. It starts with the general idea of
 image-/video-prediction~\ref{subsection::imageprediction} and what different types of prediction exist. Then it describes relevant neural network architectures, the 
 Autoencoder~\ref{subsection::autoencoder}, CNN~\ref{subsection::cnn}, RNN~\ref{subsection::rnn}, LSTM~\ref{subsection::lstm} and
 ConvLSTM~\ref{subsection::convlstm}, as they are used in the different state-of-the-art solutions. The Chapter closes with
 a description of the most important learning algorithms for neural networks, the backpropagation algorithm~\ref{subsection::backpropagation} and the more advanced 
 BPTT~\ref{subsection::bptt} algorithm.
 \\\\
 Afterwards, Chapter~\ref{section::related} gives a comprehensive overview about a variety of different state-of-the-art solutions for image-/video-prediction,
 to answer the question of what different types of image-/video-prediction algorithms exist, how they differ and what might be similar. All papers are explained
 in a way, so that the reader is able to understand the architecture and the synthetic experiments performed in the papers.
 \\\\
 Chapter~\ref{section::implementation} briefly describes the structure and usage of the re-implemented image-/video-prediction architectures or the experiments. 
 This is done, because many authors only upload
 their code base, but doesn't have a solid description of how the code is structured and how one is able to use the code, for e.g. redo the performed experiments.
 \\\\
 Afterwards, Chapter~\ref{section::experiments} gives an overview about the performed experiments, to answer the question, if
 the more advanced PredRNN~\ref{subsection::PredRNN} architecture is able to outperform the standard ConvLSTM~\ref{subsection::convlstm} in different
 state-of-the-art image-/video-prediction algorithms.\\
 It starts with the experimental setup~\ref{subsection::exp_setup},
 where the setup for the performed experiments is defined, such as the used datasets and the choice of hyperparameters. Then it concludes with the
 experimental results~\ref{subsection::exp_results}, where the results of the performed experiments are described and shown.
 \\\\
 Finally, the results are discussed in Chapter~\ref{section::discussion}. Advantages and disadvantages of the performed experiments are mentioned.
 Lastly, the thesis gets concluded and has a future outlook in Chapter~\ref{section::conclusion}.