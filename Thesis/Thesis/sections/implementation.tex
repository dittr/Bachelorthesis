% section
\section{Implementation} \label{section::implementation}
 This section describes the structure and usage of the implemented image-/video-prediction architectures (Shi et al. \cite{Shi2015}, Patraucean et al. 
 \cite{Patraucean2015}, Lotter et al. \cite{Lotter2016}), so that the reader is able to understand the given code and is able to redo the experiments and even 
 extend the code to his own needs.
  
 % subsection
 \subsection{Structure} \label{subsection::structure}
  The code is structured in a way, that everyone without specific knowledge of coding should be able to get the necessary files and results as easy as possible 
  and everyone with more experience in coding should have a nice structure to add or remove certain methods.
  \\\\
  The first important files are the setup.py and requirements.txt. Both of them are useful to install every necessary requirement to run the code on someones 
  local machine or on any machine learning cloud computing platform. This was tested during the experiments on a private computer, on 
  \href{www.floydhub.com} {Floydhub} and \href{colab.research.google.com}{Google Colab}.
  For Google Colab one needs to add a Jupyter notebook file \cite{Kluyver2016}, which is not included in the thesis.
  \\\\
  The part where all comes together is the main file in the PredNet folder, in which the main method is started. This main method controls the whole
  execution, such as initializing the network,
  initializing the optimizer (Adam \cite{Kingma2015} or RMSProp \cite{Ruder2016}) and starting the training or testing. Testing, training and validation 
  are seperated in own files, so the user has more 
  control over the methods itself (For example, the user wants to validate on more then one error, but wants to test on only one, he can simply
  add the change to the validation file, without interchanging with any other method.).
  \\
  Everything related to the actual network models is stored in the model 
  folder, which stores the three baselines and the folder for all submodules of which they consist (For example, PredNet consist of an error module, which is 
  therefore stored in model/modules/error.py.).
  Many useful methods, e.g. the choice of a certain error function can be found in the helper folder. The dataset files are stored in 
  data, as well as some simple scripts to fetch and pre-process the data. Some of this scripts are not written by the thesis author (All files written by the 
  author himself have a tag at the top of the file.). There is also the dataset folder, which holds the \href{https://pytorch.org/docs/stable/
  data.html}{PyTorch dataloader} files.
  \\
  The code is able to save and load model parameters, so one is able to continue training at a certain point, or test the network
  after training. Those network files are stored in the mdl folder.
  \\\\
  To overcome the problem of having the need to change network parameters (Such as depth, kernel sizes, padding sizes, etc.) always inside the
  code files, the implementation offers a solution to the user, where he needs to add \href{https://yaml.org/}{yml}-files which contain every necessary 
  information about the network parameters, where to store the model and the logs, using debug mode and where the dataset files are stored.\\
  The files used for the experiments in section~\ref{section::experiments} are already included, so the user is able to use them as template to create own custom
  designed networks from it. Those files are stored in the yml 
  folder.
  \\\\
  To log the training, validation and test results, the code is using Tensorboard \cite{tensorflow2015}. Those files are stored in the log folder.
  Lastly, the user is able to create the backpropagation graph from the used network. This is outputted into the graph folder. Example backpropagation graphs
  are given in the appendix~\ref{section::appendix}.
  A tree graph of the structure and a simple class and package diagram is also given in the appendix~\ref{section::appendix}.
  
 % subsection
 \subsection{Usage}
  The user starts by fetching the code using git\footnote{Github: \href{https://github.com/dittr/Bachelorthesis}{https://github.com/dittr/Bachelorthesis}}.
  Then it is recommended to use a virtual python environment, that the necessary requirements for running the code doesn't interchange with any python package
  already running on the users environment.
  \\\\
  As described in section~\ref{subsection::structure}, the first important files to look at, are requirements.txt and setup.py. Both of them have the ability to
  install the necessary requirements. For the requirements.txt file, the user simply types:
  \begin{lstlisting}[language=bash]
   python3 -m pip install --user -r requirements.txt
  \end{lstlisting}\noindent
  This simple command inserted into the command line will download all necessary frameworks and libraries for the implementation.
  Another way is to use the setup.py, where the user does:
  \begin{lstlisting}[language=bash]
   python3 setup.py install
  \end{lstlisting}\noindent
  Please note, that the setup.py will install the given code as own package, so if only downloading the requirements is necessary, this solution is preferred.
  After installing all necessary requirements, the user is able to create its own yml-files for the different network parameter.
  As written above, the used yml-files are already included, so if the user only want's to verify the experiments, one doesn't need to change the files.
  \\\\
  Then the user is able to start and run the code.
  This is done using the main.py file, where the first run should look like:
  \begin{lstlisting}[language=bash]
   python3 main.py --help
  \end{lstlisting}\noindent
  This will give the user an overview of all mandatory and optional command line parameters. An example output of this is given in the 
  appendix~\ref{section::appendix}, as well as some example listings of possible training and testing commands.
  \\\\
  The code is using Tensorboard for the error log during training and testing and also the image output during validation and testing, so to be able to use
  this tensorboard, the user should enter:
  \begin{lstlisting}[language=bash]
   tensorboard --logdir=log
  \end{lstlisting}\noindent
  If the user is using e.g. Arch Linux, this could lead to the problem, that tensorboard is not defined. An easy fix for this is:
  \begin{lstlisting}[language=bash]
   alias tensorboard='python3 -m tensorboard.main'
  \end{lstlisting}\noindent